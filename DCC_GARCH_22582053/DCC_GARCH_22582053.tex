\documentclass[12pt,preprint, authoryear]{elsarticle}

\usepackage{lmodern}
%%%% My spacing
\usepackage{setspace}
\setstretch{1.2}
\DeclareMathSizes{12}{14}{10}{10}

% Wrap around which gives all figures included the [H] command, or places it "here". This can be tedious to code in Rmarkdown.
\usepackage{float}
\let\origfigure\figure
\let\endorigfigure\endfigure
\renewenvironment{figure}[1][2] {
    \expandafter\origfigure\expandafter[H]
} {
    \endorigfigure
}

\let\origtable\table
\let\endorigtable\endtable
\renewenvironment{table}[1][2] {
    \expandafter\origtable\expandafter[H]
} {
    \endorigtable
}


\usepackage{ifxetex,ifluatex}
\usepackage{fixltx2e} % provides \textsubscript
\ifnum 0\ifxetex 1\fi\ifluatex 1\fi=0 % if pdftex
  \usepackage[T1]{fontenc}
  \usepackage[utf8]{inputenc}
\else % if luatex or xelatex
  \ifxetex
    \usepackage{mathspec}
    \usepackage{xltxtra,xunicode}
  \else
    \usepackage{fontspec}
  \fi
  \defaultfontfeatures{Mapping=tex-text,Scale=MatchLowercase}
  \newcommand{\euro}{€}
\fi

\usepackage{amssymb, amsmath, amsthm, amsfonts}

\def\bibsection{\section*{References}} %%% Make "References" appear before bibliography


\usepackage[round]{natbib}

\usepackage{longtable}
\usepackage[margin=2.3cm,bottom=2cm,top=2.5cm, includefoot]{geometry}
\usepackage{fancyhdr}
\usepackage[bottom, hang, flushmargin]{footmisc}
\usepackage{graphicx}
\numberwithin{equation}{section}
\numberwithin{figure}{section}
\numberwithin{table}{section}
\setlength{\parindent}{0cm}
\setlength{\parskip}{1.3ex plus 0.5ex minus 0.3ex}
\usepackage{textcomp}
\renewcommand{\headrulewidth}{0.2pt}
\renewcommand{\footrulewidth}{0.3pt}

\usepackage{array}
\newcolumntype{x}[1]{>{\centering\arraybackslash\hspace{0pt}}p{#1}}

%%%%  Remove the "preprint submitted to" part. Don't worry about this either, it just looks better without it:
\makeatletter
\def\ps@pprintTitle{%
  \let\@oddhead\@empty
  \let\@evenhead\@empty
  \let\@oddfoot\@empty
  \let\@evenfoot\@oddfoot
}
\makeatother

 \def\tightlist{} % This allows for subbullets!

\usepackage{hyperref}
\hypersetup{breaklinks=true,
            bookmarks=true,
            colorlinks=true,
            citecolor=blue,
            urlcolor=blue,
            linkcolor=blue,
            pdfborder={0 0 0}}


% The following packages allow huxtable to work:
\usepackage{siunitx}
\usepackage{multirow}
\usepackage{hhline}
\usepackage{calc}
\usepackage{tabularx}
\usepackage{booktabs}
\usepackage{caption}


\newenvironment{columns}[1][]{}{}

\newenvironment{column}[1]{\begin{minipage}{#1}\ignorespaces}{%
\end{minipage}
\ifhmode\unskip\fi
\aftergroup\useignorespacesandallpars}

\def\useignorespacesandallpars#1\ignorespaces\fi{%
#1\fi\ignorespacesandallpars}

\makeatletter
\def\ignorespacesandallpars{%
  \@ifnextchar\par
    {\expandafter\ignorespacesandallpars\@gobble}%
    {}%
}
\makeatother

\newenvironment{CSLReferences}[2]{%
}

\urlstyle{same}  % don't use monospace font for urls
\setlength{\parindent}{0pt}
\setlength{\parskip}{6pt plus 2pt minus 1pt}
\setlength{\emergencystretch}{3em}  % prevent overfull lines
\setcounter{secnumdepth}{5}

%%% Use protect on footnotes to avoid problems with footnotes in titles
\let\rmarkdownfootnote\footnote%
\def\footnote{\protect\rmarkdownfootnote}
\IfFileExists{upquote.sty}{\usepackage{upquote}}{}

%%% Include extra packages specified by user

%%% Hard setting column skips for reports - this ensures greater consistency and control over the length settings in the document.
%% page layout
%% paragraphs
\setlength{\baselineskip}{12pt plus 0pt minus 0pt}
\setlength{\parskip}{12pt plus 0pt minus 0pt}
\setlength{\parindent}{0pt plus 0pt minus 0pt}
%% floats
\setlength{\floatsep}{12pt plus 0 pt minus 0pt}
\setlength{\textfloatsep}{20pt plus 0pt minus 0pt}
\setlength{\intextsep}{14pt plus 0pt minus 0pt}
\setlength{\dbltextfloatsep}{20pt plus 0pt minus 0pt}
\setlength{\dblfloatsep}{14pt plus 0pt minus 0pt}
%% maths
\setlength{\abovedisplayskip}{12pt plus 0pt minus 0pt}
\setlength{\belowdisplayskip}{12pt plus 0pt minus 0pt}
%% lists
\setlength{\topsep}{10pt plus 0pt minus 0pt}
\setlength{\partopsep}{3pt plus 0pt minus 0pt}
\setlength{\itemsep}{5pt plus 0pt minus 0pt}
\setlength{\labelsep}{8mm plus 0mm minus 0mm}
\setlength{\parsep}{\the\parskip}
\setlength{\listparindent}{\the\parindent}
%% verbatim
\setlength{\fboxsep}{5pt plus 0pt minus 0pt}



\begin{document}



\begin{frontmatter}  %

\title{MV-Volatility modeling applied to local equities}

% Set to FALSE if wanting to remove title (for submission)




\author[Add1]{Austin Byrne}
\ead{22582053@sun.ac.za}





\address[Add1]{Stellenbosch University}

\cortext[cor]{Corresponding author: Austin Byrne}

\begin{abstract}
\small{
The main aim of this study is to evaluate the volatility structure of
local equities. This analysis will be achieved by making use of
DCC-GARCH multivariate volatility model. Through this analysis I
indetify the most volatile index/sector to be the Resource sector, while
the ALSI and Industrial sector to hold the lowest volatility.
}
\end{abstract}

\vspace{1cm}





\vspace{0.5cm}

\end{frontmatter}

\setcounter{footnote}{0}



%________________________
% Header and Footers
%%%%%%%%%%%%%%%%%%%%%%%%%%%%%%%%%
\pagestyle{fancy}
\chead{}
\rhead{}
\lfoot{}
\rfoot{\footnotesize Page \thepage}
\lhead{}
%\rfoot{\footnotesize Page \thepage } % "e.g. Page 2"
\cfoot{}

%\setlength\headheight{30pt}
%%%%%%%%%%%%%%%%%%%%%%%%%%%%%%%%%
%________________________

\headsep 35pt % So that header does not go over title




\hypertarget{introduction}{%
\section{\texorpdfstring{Introduction
\label{Introduction}}{Introduction }}\label{introduction}}

When dealing with financial markets and portfolio creation it is
paramount for an investor or risk managers to understand the risk
involved in their investment decisions. Within financial markets,
volatility is often evaluated as a measure of uncertainty and risk.
Thus, through evaluating the volatility structure of indexes we are able
to understand the risk structure involved with certain investment
decisions. Furthermore, by evaluating the dynamic correlations between
indexes we are able to better diversify a portfolio and reduce risk. The
main objective of this project is to understand the multivariate
volatility structure of local equities and thus understand the relative
risk associated with certain investment decisions.

The use of Dynamic Conditional Correlation (DCC) GARCH models to
evaluate market dynamics and dynamic correlations has become more
frequent in portfolio creation and financial theory. Boudt, Danielsson
\& Laurent (\protect\hyperlink{ref-boudt2013robust}{2013}) make use of
an extension of a DCC GARCH model to forecasting the covariance matrix
of the daily EUR/USD and Yen/USD return series
(\protect\hyperlink{ref-boudt2013robust}{Boudt \emph{et al.}, 2013}).
Additionally, Shiferaw (\protect\hyperlink{ref-shiferaw2019time}{2019})
run a multivariate DCC-GARCH model to evaluate time varying correlation
between energy price dynamics and agricultural commodity to name a few
(\protect\hyperlink{ref-shiferaw2019time}{Shiferaw, 2019}).

Understanding this topic can be vitally important for investors and
portfolio managers when making investment decisions. Realizing the risk
exposure a portfolio is inherently exposed to is crucial when making
investment decisions. Thus, by modeling the volatility of local equities
an investor can make more informed investment decisions based on the
intended level of risk the investors wants to take on. This project aims
to achieve this analysis through a Dynamic Conditional Correlation (DCC)
GARCH model capturing the market dynamics and volatility structures of
local sectors and indexes. This model is renowned for its proficiency in
assessing multivariate time series data, particularly its capability to
capture the evolving correlations between different financial time
series.

The results of this study find the Resource sector to be the most
volatile with the ALSI and Industrial sector being the least volatile.

The remainder of this paper is ordered as follows: The next section
covers the Data and Methodology, followed by the Results section,
followed by the discussion section and lastly the Conclusion.

\hypertarget{data-methodology}{%
\section{\texorpdfstring{Data Methodology
\label{Data Methodology}}{Data Methodology }}\label{data-methodology}}

Throughout this paper multivariate volatility modeling techniques will
be used with the focus on understanding the dynamic relationships and
co-movements of volatility across multiple sectors/indexes within the
local equities market.

Of the MV-volatility modeling techniques that there are available, I
will be utilizing the DCC-GARCH model. My aim is to use DCC models in
the multivariate volatility analysis as these models are simpler and
relax the constraint of a fixed correlation structure which is assumed
by the CCC model, which allows for estimates of time varying
correlation. I will be conducting the DCC-GARCH volatility model in R
statistical package.

The data I am using holds the returns for the following indexes/sectors,
ALSI, Financial Sector, Industrial Sector, jsapy index and the Resource
Sector.

\hypertarget{variation-in-index-returns}{%
\subsection{Variation in index
returns}\label{variation-in-index-returns}}

Figure \ref{Figure1} below plots the time series analysis of each
indexes returns. The importance of this plot for this analysis lies in
it's ability to portray the variance structure in each indexes returns.
By visualizing the return structure of an index through time we are able
to identify the most volatile and thus more risky indexes.

While holding the y-axis fixed, figure \ref{Figure1} shows that the most
volatile index/sector is that of the Resource sector. While the ALSI and
Industrial sector appear to be the least volatile and thus lowest risk
investment opportunities.

To further evaluate the return structure of the indexes/sectors in
question figure \ref{Figure2} analyses the distribution of returns
through a histogram plot.

\begin{figure}[H]

{\centering \includegraphics{DCC_GARCH_22582053_files/figure-latex/Figure1-1} 

}

\caption{Time Series Analysis of Returns \label{Figure1}}\label{fig:Figure1}
\end{figure}

\hypertarget{analyzing-the-distribution-of-returns}{%
\subsection{Analyzing the Distribution of
Returns}\label{analyzing-the-distribution-of-returns}}

Through figure \ref{Figure1} we have identified the resource sector as
the most volatile and thus most risky investment option, while the ALSI
and industrial sector to be the least volatile. To build on this
analysis I run a histogram plot of the return structure. Through this
plot we are able to identify any fat tails or skewness in returns as
well as identify the index/sector with the widest variation in returns.

The results of figure \ref{Figure2} below are inline with that found in
figure \ref{Figure1} above. The resource sector once again appears to be
the most volatile with the ALSI and industrial sector appearing to be
the least volatile.

Following the analysis of figures \ref{Figure1} and \ref{Figure2} we
have now obtained a better understanding of the return structure and
volatility of the indexes/sectors in question. Thus, we are now able to
proceed to the multivariate volatility analysis using the DCC-GARCH
model. However, Before I fit any GARCH models to my data I need evaluate
the presence of any ARCH effects. If there are no ARCH effects present,
it does not make sense to run any GARCH models for this volatility
analysis. To test for ARCH effects I perform the McLeod-Li test. This
test checks for auto regressive conditional heteroskedasticity (ARCH) in
the time series data (\protect\hyperlink{ref-wang2005testing}{Wang, Van
Gelder, Vrijling \& Ma, 2005}). If present the p-values will be very
close to 0, indicating that the volatility within the variables changes
over time. Thus, it is viable to use GARCH models. This test is done in
the next section.

\begin{figure}[H]

{\centering \includegraphics{DCC_GARCH_22582053_files/figure-latex/Figure2-1} 

}

\caption{Distribution of Returns \label{Figure2}}\label{fig:Figure2}
\end{figure}

\hypertarget{testing-for-arch-effects}{%
\subsection{Testing for ARCH Effects}\label{testing-for-arch-effects}}

The results in table 2.1 below provide evidence of ARCH effects. All
p-values are close to zero which indicates that there is statistical
significance indicating that the volatility between the variables change
over time. Thus, it is viable to make use of a GARCH model to conduct
the volatility analysis.

In the next section I will make use of the DCC-GARCH model to plot the
Dynamic Conditional Correlation plots and report the results.

\begin{table}

\caption{\label{tab:unnamed-chunk-12}McLeod-Li Test Results for Autoregressive Conditional Heteroskedasticity}
\centering
\begin{tabular}[t]{lrr}
\toprule
Test & TestStatistic & PValue\\
\midrule
Q(m) of squared series (LM test) & 44.88386 & 0.0000023\\
Rank-based Test & 108.25550 & 0.0000000\\
Q\_k(m) of squared series & 341.50090 & 0.0001048\\
Robust Test (5\%) & 321.42970 & 0.0015293\\
\bottomrule
\end{tabular}
\end{table}

\hypertarget{results}{%
\section{\texorpdfstring{Results
\label{Results}}{Results }}\label{results}}

In the following section I plot a volatility comparison plot with the
purpose of further understanding which index/sector inherently is
exposed to the most volatility. Furthermore, I run the Dynamic
Conditional Correlation plots for each index/sector.

\hypertarget{volatility-comparisson}{%
\subsection{Volatility Comparisson}\label{volatility-comparisson}}

Figure \ref{Figure3} below represents the volatility comparison plot.
Through evaluating this plot it is once again evident that the resources
sector is consistently the most volatile index/sector. Furthermore, post
2020 their is a significant spike in the volatility in the Financials
sector and the jsapy, thus suggesting the Financials Sector and jsapy
index are the most sensitive to large shocks in the economy, such as the
COVID-19 pandemic. Additionally, the Industrial sector remains the least
volatile along with the ALSI.

Now that I have further strengthened my understanding of the nature of
volatility for each index/sector, I will now focus my analysis on the
conditional correlations in the next section.

\begin{figure}[H]

{\centering \includegraphics{DCC_GARCH_22582053_files/figure-latex/Figure3-1} 

}

\caption{Volatility comparisson plot \label{Figure3}}\label{fig:Figure3}
\end{figure}

\hypertarget{dynamic-conditional-correlations}{%
\subsection{Dynamic Conditional
Correlations}\label{dynamic-conditional-correlations}}

In this section I plot the Dynamic Conditional Correlation plots for
each index/sector. Through these plots we are able to identify which
index's/sectors are correlated over time. This is important information
when creating optimal portfolios, ensuring diversification and reducing
risk for portfolio managers and investors alike.

Figure \ref{Figure4} represents the Dynamic Conditional Correlations
plot for both the Resources sector and jsapy index. Figure \ref{Figure5}
represents the Dynamic Conditional Correlations plot for the Financial
and Industrial sector. Lastly, Figure \ref{Figure6} represents the
Dynamic Conditional Correlations plot for the ALSI.

Figure \ref{Figure4} illustrates the Resource sector has a high dynamic
correlation with the ALSI with a correlation above 0.5 throughout the
entire time frame. Furthermore, the Resources sector has the lowest
dynamic correlation with the jsapy index. Additionally, Figure
\ref{Figure4} illustrates that the jsapy index has a high dynamic
correlation with the Financial sector and lowest dynamic correlation
with the Resource sector.

Figure \ref{Figure5} illustrates that the Financial sector holds a high
dynamic correlation with the jsapy index, ALSI and the Industrial
sector. While holding a low dynamic correlation with the Resource
sector. Furthermore, Figure \ref{Figure5} illustrates that the
Industrial Sector holds a strong dynamic correlation with the ALSI and
holds its lowest dynamic correlation with the Resource sector and jsapy.

Lastly, Figure\ref{Figure6} illustrates that the ALSI holds a strong
dynamic correlation with the Industrial and Resource sector while
holding its lowest dynamic correlation with the jsapy index.

\begin{figure}[H]

{\centering \includegraphics{DCC_GARCH_22582053_files/figure-latex/Figure4-1} 

}

\caption{Dynamic Conditional Correlations: Resources and jsapy \label{Figure4}}\label{fig:Figure4}
\end{figure}

\begin{figure}[H]

{\centering \includegraphics{DCC_GARCH_22582053_files/figure-latex/Figure5-1} 

}

\caption{Dynamic Conditional Correlations: Financials and Industrials \label{Figure5}}\label{fig:Figure5}
\end{figure}

\begin{figure}[H]

{\centering \includegraphics{DCC_GARCH_22582053_files/figure-latex/Figure6-1} 

}

\caption{Dynamic Conditional Correlations: ALSI \label{Figure6}}\label{fig:Figure6}
\end{figure}

\hypertarget{discussion}{%
\section{\texorpdfstring{Discussion
\label{Discussion}}{Discussion }}\label{discussion}}

In the above analysis I have evaluated the overall volatility of each
index/sector and compared the results with each other. Furthermore, I
have evaluated the dynamic correlations between each index/sector. With
these results one can identify which indexes/sector hold the highest
risk and which indexes/sectors to combine in a portfolio to induce
diversification and risk mitigation.

Through my analysis it is evident that the Resource sector holds the
highest volatility and thus highest risk while the ALSI and Industrial
sector hold the lowest volatility and thus lowest source of risk. Thus
if you were attempting to build a low risk portfolio you would hold a
high proportion of the ALSI and Industrial sector. If this were the case
and you intended to impose some diversification into your portfolio we
could evaluate the Dynamic Conditional Correlation plots for the ALSI
and Industrial sector.

With respect to the ALSI investing in the jsapy index would provide for
increased diversification in your portfolio. This addition to the
portfolio could be warranted as the jsapy index holds average volatility
and thus average risk. Therefore, by introducing the jsapy to the
portfolio we are increasing the diversification while not exposing the
portfolio to a great deal of risk. With respect to the Industrial Sector
investing in the Resource sector would induce diversification although
increasing risk through the highly volatile Resource Sector.

Therefore, a portfolio that holds a large portion in the ALSI and
Industrial sector with slight investment in the jsapy and even slighter
investment in the Resource sector will provide for a low risk,
diversified investment portfolio.

\hypertarget{conclusion}{%
\section{\texorpdfstring{Conclusion
\label{Conclusion}}{Conclusion }}\label{conclusion}}

To conclude, throughout this study I evaluate the return structure and
volatility of the ALSI, Financial Sector, Industrial Sector, jsapy index
and the Resource Sector. Furthermore, by making use of the DCC-GARCH
volatility model I was able to establish the dynamic conditional
correlation for each index/sector in question.

The results found the Resource sector to be the most volatile and thus
most risky while finding the ALSI and Industrial sector to be the least
volatile and thus lowest risk options of the available indexes/sectors.
Through evaluating the respective dynamic correlations it is found that
the most optimal portfolio construction could potentially be, a
portfolio that holds a large portion in the ALSI and Industrial sector
with slight investment in the jsapy and even slighter investment in the
Resource sector as this portfolio would induce a low risk and well
diversified portfolio.

The limitations to the study are that we only focus on the ALSI,
Financial Sector, Industrial Sector, jsapy index and the Resource
Sector, thus the results are only limited to these indexes/sectors.

\newpage

\hypertarget{references}{%
\section*{References}\label{references}}
\addcontentsline{toc}{section}{References}

\hypertarget{refs}{}
\begin{CSLReferences}{1}{0}
\leavevmode\vadjust pre{\hypertarget{ref-boudt2013robust}{}}%
Boudt, K., Danielsson, J. \& Laurent, S. 2013. Robust forecasting of
dynamic conditional correlation GARCH models. \emph{International
Journal of Forecasting}. 29(2):244--257.

\leavevmode\vadjust pre{\hypertarget{ref-shiferaw2019time}{}}%
Shiferaw, Y.A. 2019. Time-varying correlation between agricultural
commodity and energy price dynamics with bayesian multivariate DCC-GARCH
models. \emph{Physica A: Statistical Mechanics and Its Applications}.
526:120807.

\leavevmode\vadjust pre{\hypertarget{ref-wang2005testing}{}}%
Wang, W., Van Gelder, P.M., Vrijling, J. \& Ma, J. 2005. Testing and
modelling autoregressive conditional heteroskedasticity of streamflow
processes. \emph{Nonlinear processes in Geophysics}. 12(1):55--66.

\end{CSLReferences}

\bibliography{Tex/ref}





\end{document}
